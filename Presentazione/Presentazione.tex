\documentclass[10pt]{beamer}

\usetheme{Madrid}
\usecolortheme{wolverine}
\setbeamertemplate{navigation symbols}{}

\setbeamercovered{transparent}
\setbeamertemplate{headline}{}


\title{Bayan-Unjuul's Analysis}
\author{Alberto Tarroni}


\begin{document}

\maketitle
\begin{frame}{Index}
\tableofcontents
\end{frame}

\section{Introduction}
\begin{frame}
\frametitle{Contenuto della presentazione}
\begin{itemize}
	\item<1-> AAAA
	\item<2-> bbbb
	\item<3-> cccc
\end{itemize}
\end{frame}

\section{Winter}

\begin{frame}
	\begin{columns}
		\column{0.5\textwidth}Rimetti i pesci nel laghetto
		\column{0.5\textwidth}rimetti i pesciolini nel laghetto
	\end{columns}
\end{frame}


\section{Summer}
\begin{frame}
	\begin{columns}
		\column{0.5\textwidth}Rimetti i pesci nel laghetto
		\column{0.5\textwidth}rimetti i pesciolini nel laghetto
	\end{columns}
\end{frame}


\section{Spring}
\begin{frame}
	\begin{columns}
		\column{0.5\textwidth}Rimetti i pesci nel laghetto
		\column{0.5\textwidth}rimetti i pesciolini nel laghetto
	\end{columns}
\end{frame}


\section{Conclusions}
\begin{frame}
	\begin{columns}
		\column{0.5\textwidth}Rimetti i pesci nel laghetto
		\column{0.5\textwidth}rimetti i pesciolini nel laghetto
	\end{columns}
\end{frame}


\end{document}

